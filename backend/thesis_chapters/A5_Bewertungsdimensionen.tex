% Anhang C: Bewertungsdimensionen und Kriterien im Detail

\addchap{Anhang C: Bewertungsdimensionen und Kriterien}
\label{app:bewertungsdimensionen}

Dieser Anhang dokumentiert die vollständigen Bewertungsdimensionen und -kriterien des KI-Governance-Bewertungsframeworks mit den zugehörigen Reifegradstufen-Indikatoren.

\section*{D1 -- Risikomanagement (Art.~9 EU AI Act)}

\begin{table}[htbp]
\centering
\small
\renewcommand{\arraystretch}{1.2}
\begin{tabular}{|p{1cm}|p{3cm}|p{2.5cm}|p{2.5cm}|p{2.5cm}|}
\hline
\textbf{ID} & \textbf{Kriterium} & \textbf{Stufe 1} & \textbf{Stufe 3} & \textbf{Stufe 5} \\
\hline
D1.1 & KI-spezifisches Risikomgmt.-System & Nicht vorhanden & Dokumentiert u. standardisiert & Kont. optimiert, Best-Practice \\
\hline
D1.2 & Syst. Risikoidentifikation & Ad-hoc & Standardisierter Prozess & Prädiktive Analyse \\
\hline
D1.3 & Risikobewertungsmethodik & Keine Methodik & Qual./quant. Bewertung & KRI/KCI-basiert \\
\hline
D1.4 & Risikobehandlung & Reaktiv & Definierte Maßnahmen & Automatisierte Steuerung \\
\hline
D1.5 & Testverfahren & Keine Tests & Regelmäßige Tests & Continuous Testing \\
\hline
D1.6 & Kont. Aktualisierung & Statisch & Periodische Reviews & Real-time Updates \\
\hline
\end{tabular}
\caption{Bewertungskriterien D1 -- Risikomanagement (Auszug Stufen 1, 3, 5)}
\label{tab:d1-kriterien}
\end{table}

\section*{D2 -- Data Governance (Art.~10 EU AI Act)}

\begin{table}[htbp]
\centering
\small
\renewcommand{\arraystretch}{1.2}
\begin{tabular}{|p{1cm}|p{3cm}|p{2.5cm}|p{2.5cm}|p{2.5cm}|}
\hline
\textbf{ID} & \textbf{Kriterium} & \textbf{Stufe 1} & \textbf{Stufe 3} & \textbf{Stufe 5} \\
\hline
D2.1 & Datenqualitätskriterien & Keine Standards & Dokumentierte Standards & Automatisierte Qualitätssicherung \\
\hline
D2.2 & Bias-Erkennung & Nicht adressiert & Systematische Tests & Continuous Monitoring \\
\hline
D2.3 & Data Lineage & Nicht dokumentiert & Nachvollziehbar dokumentiert & Automatisierte Lineage \\
\hline
D2.4 & DSGVO-Integration & Getrennt behandelt & Integrierter Prozess & Unified Governance \\
\hline
D2.5 & Kont. Datenqualität & Einmalig & Periodische Prüfung & Real-time Monitoring \\
\hline
\end{tabular}
\caption{Bewertungskriterien D2 -- Data Governance (Auszug Stufen 1, 3, 5)}
\label{tab:d2-kriterien}
\end{table}

\section*{D3 -- Dokumentation (Art.~11--12 EU AI Act)}

\begin{table}[htbp]
\centering
\small
\renewcommand{\arraystretch}{1.2}
\begin{tabular}{|p{1cm}|p{3cm}|p{2cm}|p{2.5cm}|p{2.5cm}|}
\hline
\textbf{ID} & \textbf{Kriterium} & \textbf{Stufe 1} & \textbf{Stufe 3} & \textbf{Stufe 5} \\
\hline
D3.1 & Techn. Dokumentation & Lückenhaft & Vollständig, aktuell & Auto-generiert, validiert \\
\hline
D3.2 & Logging & Kein Logging & Standardisiertes Logging & Umfassendes Audit-Trail \\
\hline
D3.3 & Versionierung & Keine & Git-basiert & Vollständige Provenance \\
\hline
D3.4 & Konformitätsnachweise & Nicht vorhanden & Manuell erstellt & Semi-automatisiert \\
\hline
D3.5 & Dokumentationsmgmt. & Verstreut & Zentral verwaltet & Integriertes DMS \\
\hline
\end{tabular}
\caption{Bewertungskriterien D3 -- Dokumentation (Auszug Stufen 1, 3, 5)}
\label{tab:d3-kriterien}
\end{table}

\section*{D4 -- Transparenz (Art.~13 EU AI Act)}

\begin{table}[htbp]
\centering
\small
\renewcommand{\arraystretch}{1.2}
\begin{tabular}{|p{1cm}|p{3cm}|p{2cm}|p{2.5cm}|p{2.5cm}|}
\hline
\textbf{ID} & \textbf{Kriterium} & \textbf{Stufe 1} & \textbf{Stufe 3} & \textbf{Stufe 5} \\
\hline
D4.1 & Erklärbarkeit & Nicht adressiert & Erklärbarkeitsmethoden eingesetzt & Multi-Level-Erklärungen \\
\hline
D4.2 & Informationspflichten & Nicht erfüllt & Dokumentiert, bereitgestellt & Nutzergerecht aufbereitet \\
\hline
D4.3 & Fähigkeiten/Grenzen & Nicht kommuniziert & Intern dokumentiert & Proaktiv kommuniziert \\
\hline
D4.4 & KI-Kennzeichnung & Nicht vorhanden & Standardisiert & Automatisiert \\
\hline
D4.5 & Prozess-Transparenz & Intransparent & Dokumentiert & Stakeholder-Dashboard \\
\hline
\end{tabular}
\caption{Bewertungskriterien D4 -- Transparenz (Auszug Stufen 1, 3, 5)}
\label{tab:d4-kriterien}
\end{table}

\section*{D5 -- Menschliche Aufsicht (Art.~14 EU AI Act)}

\begin{table}[htbp]
\centering
\small
\renewcommand{\arraystretch}{1.2}
\begin{tabular}{|p{1cm}|p{3cm}|p{2cm}|p{2.5cm}|p{2.5cm}|}
\hline
\textbf{ID} & \textbf{Kriterium} & \textbf{Stufe 1} & \textbf{Stufe 3} & \textbf{Stufe 5} \\
\hline
D5.1 & Aufsichtsdesign & Nicht vorgesehen & HITL/HOTL definiert & Adaptives Aufsichtsdesign \\
\hline
D5.2 & Qualifikation & Nicht sichergestellt & Schulungsprogramm & Zertifizierung, kont. Weiterbildung \\
\hline
D5.3 & Intervention & Keine Mechanismen & Definierte Prozesse & Automatisierte Eskalation \\
\hline
D5.4 & Automation-Bias-Prävention & Nicht adressiert & Awareness-Training & Kalibrierte Entscheidungshilfen \\
\hline
D5.5 & Aufsichts-Review & Nicht dokumentiert & Regelmäßige Reviews & Datenbasierte Analyse \\
\hline
\end{tabular}
\caption{Bewertungskriterien D5 -- Menschliche Aufsicht (Auszug Stufen 1, 3, 5)}
\label{tab:d5-kriterien}
\end{table}

\section*{D6 -- Technische Robustheit (Art.~15 EU AI Act)}

\begin{table}[htbp]
\centering
\small
\renewcommand{\arraystretch}{1.2}
\begin{tabular}{|p{1cm}|p{3cm}|p{2cm}|p{2.5cm}|p{2.5cm}|}
\hline
\textbf{ID} & \textbf{Kriterium} & \textbf{Stufe 1} & \textbf{Stufe 3} & \textbf{Stufe 5} \\
\hline
D6.1 & Genauigkeitsmetriken & Keine definiert & Standard-Metriken & Kontextspezifische Schwellen \\
\hline
D6.2 & Robustheit & Nicht getestet & Adversariale Tests & Continuous Robustness Testing \\
\hline
D6.3 & Cybersicherheit & Basis-Schutz & Dedizierte Maßnahmen & Security-by-Design \\
\hline
D6.4 & Drift-Erkennung & Nicht implementiert & Periodisches Monitoring & Real-time Detection \\
\hline
D6.5 & Fallback-Mechanismen & Nicht vorhanden & Definierte Fallbacks & Graceful Degradation \\
\hline
\end{tabular}
\caption{Bewertungskriterien D6 -- Technische Robustheit (Auszug Stufen 1, 3, 5)}
\label{tab:d6-kriterien}
\end{table}

Die Indikatoren für die Reifegradstufen~2 (Managed) und~4 (Measured) sind als qualitative Übergangsstufen entlang der drei Entwicklungsachsen (Formalisierungsgrad, Organisationale Verankerung, Steuerungsmodus) definiert: Stufe~2 beschreibt den Übergang von ad~hoc zu dokumentierten Maßnahmen, Stufe~4 den Übergang von standardisierten zu messungsbasierten Prozessen (vgl.\ Abschnitt~\ref{subsec:reifegradmodell}). Tabelle~\ref{tab:stufen-2-4-beispiele} illustriert diese Übergangsstufen exemplarisch anhand ausgewählter Kriterien.

\begin{table}[htbp]
\centering
\small
\renewcommand{\arraystretch}{1.2}
\begin{tabular}{|p{1cm}|p{2.5cm}|p{4.5cm}|p{4.5cm}|}
\hline
\textbf{ID} & \textbf{Kriterium} & \textbf{Stufe 2 (Managed)} & \textbf{Stufe 4 (Measured)} \\
\hline
D1.2 & Syst. Risikoidentifikation & Risiken werden projektbezogen erfasst, jedoch ohne einheitliche Methodik oder Taxonomie & Standardisierter Prozess wird durch quantitative KRI überwacht; Abdeckungsgrad wird gemessen \\
\hline
D2.2 & Bias-Erkennung & Bias-Prüfung erfolgt anlassbezogen bei bekannten Risiken, nicht systematisch & Systematische Tests werden durch Fairness-Metriken ergänzt; Ergebnisse fließen in Steuerungsentscheidungen ein \\
\hline
D5.1 & Aufsichtsdesign & Menschliche Aufsicht ist vorgesehen, aber informell geregelt (keine definierten HITL/HOTL-Rollen) & HITL/HOTL-Prozesse sind definiert; Aufsichtsentscheidungen werden quantitativ ausgewertet und optimiert \\
\hline
\end{tabular}
\caption{Exemplarische Indikatoren für die Übergangsstufen 2 (Managed) und 4 (Measured)}
\label{tab:stufen-2-4-beispiele}
\end{table}

Die vollständigen Indikatoren für alle 31~Kriterien auf allen fünf Stufen sind im Self-Assessment-Tool des Prototyps (Anhang~\ref{app:prototyp}) implementiert.
