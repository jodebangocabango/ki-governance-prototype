% Vorlage für WABs der Provadis Hochschule
% basiert auf "Universität Ulm Praktikumsbericht Vorlage"
% von Max Sch.
% CC BY 4.0
%
% Abstract.tex

\addcontentsline{toc}{chapter}{Abstract}
\begin{abstract}
Mit dem EU AI Act (Verordnung (EU) 2024/1689) gelten erstmals verbindliche Governance-Anforderungen für KI-Systeme. Bestehende Rahmenwerke -- darunter das NIST AI RMF und die ISO/IEC~42001 -- erfüllen nicht gleichzeitig vier Kriterien: EU-AI-Act-Spezifität, operationalisierbare Bewertungskriterien, abgestufte Reifegradlogik und prototypische Praxisvalidierung.

Die vorliegende Masterarbeit adressiert diese Lücke mit einem Design-Science-Research-Ansatz. Das entwickelte Bewertungsframework überführt die Art.~9--15 des EU AI Acts in sechs Governance-Dimensionen: Risikomanagement, Data Governance, Dokumentation, Transparenz, Menschliche Aufsicht und Technische Robustheit. Eine qualitative Inhaltsanalyse nach Mayring identifiziert aus 187~kodierten Textstellen 31~Bewertungskriterien und zwei induktiv gewonnene Querschnittskategorien (Organisationale Verankerung, Kompetenzentwicklung), die über den Normtext hinausgehen. Ein fünfstufiges Reifegradmodell (Initial -- Optimizing) definiert Stufe~3 als regulatorische Compliance-Baseline. Der webbasierte Prototyp trennt deterministische Bewertungslogik von KI-gestützter Unterstützung (RAG) und erprobt eine Architektur, die für regulatorisch sensible KI-Anwendungen relevant sein kann.

Die vierstufige Evaluation bestätigt regulatorische Vollständigkeit (100\,\% Abdeckung) und hohe Konsistenz (MW~4,5/5). Bei Verständlichkeit (MW~3,6/5) und Praxistauglichkeit (MW~3,5/5) zeigt sich ein Zielgruppeneffekt: Evaluierende mit Compliance-Erfahrung bewerten das Framework positiver als technische Fachleute. Die Stichprobe ($n = 8$) liefert qualitative Hinweise, jedoch keine statistisch generalisierbaren Ergebnisse.

Die Arbeit leistet einen Beitrag zur Operationalisierung der Principle-Practice-Gap in der KI-Governance und stellt ein strukturiertes Instrument bereit, mit dem Organisationen ihren Vorbereitungsstand auf die ab August 2026 geltenden Hochrisiko-Anforderungen systematisch einschätzen können.

\medskip
\noindent\textbf{Schlagwörter:} KI-Governance, EU AI Act, Bewertungsframework, Reifegradmodell, Design Science Research, Hochrisiko-KI-Systeme

\bigskip
\selectlanguage{english}
\noindent\textbf{Abstract}

\medskip
\noindent With the EU AI Act (Regulation (EU) 2024/1689), binding governance requirements for AI systems apply for the first time. Existing frameworks---including the NIST AI RMF and ISO/IEC~42001---do not simultaneously meet four criteria: EU AI Act specificity, operationalizable assessment criteria, graduated maturity logic, and prototypical practical validation.

This master's thesis addresses this gap using a Design Science Research approach. The developed assessment framework translates Articles~9--15 of the EU AI Act into six governance dimensions: Risk Management, Data Governance, Documentation, Transparency, Human Oversight, and Technical Robustness. A qualitative content analysis following Mayring identifies 31~assessment criteria and two inductively derived cross-cutting categories (Organizational Anchoring, Competency Development) from 187~coded text passages that extend beyond the normative text. A five-level maturity model (Initial--Optimizing) defines Level~3 as the regulatory compliance baseline. The web-based prototype separates deterministic assessment logic from AI-supported assistance (RAG) and explores an architecture relevant to regulatory-sensitive AI applications.

The four-stage evaluation confirms regulatory completeness (100\,\% coverage) and high consistency (M~=~4.5/5). For comprehensibility (M~=~3.6/5) and practical applicability (M~=~3.5/5), a target group effect emerges: evaluators with compliance experience rate the framework more positively than technical experts. The sample ($n = 8$) provides qualitative insights but no statistically generalizable results.

This thesis contributes to operationalizing the principle-practice gap in AI governance and provides a structured instrument enabling organizations to systematically assess their preparedness for the high-risk requirements taking effect from August~2026.

\medskip
\noindent\textbf{Keywords:} AI Governance, EU AI Act, Assessment Framework, Maturity Model, Design Science Research, High-Risk AI Systems
\selectlanguage{ngerman}
\end{abstract}
