% Anhang B: Vollständige Anforderungsmatrix

\addchap{Anhang B: Anforderungsmatrix (Kategorien $\rightarrow$ Requirements)}
\label{app:anforderungsmatrix}

Die folgende Matrix dokumentiert die systematische Überführung der in der qualitativen Inhaltsanalyse identifizierten Kategorien in die Designanforderungen des Bewertungsframeworks.

\begin{table}[htbp]
\centering
\small
\renewcommand{\arraystretch}{1.2}
\begin{tabular}{|p{1.5cm}|p{3cm}|p{2.5cm}|p{3.5cm}|p{2.5cm}|}
\hline
\textbf{Kat.} & \textbf{Subkategorie} & \textbf{Dimension} & \textbf{Bewertungs\-kriterium} & \textbf{Requirement} \\
\hline
K1.1 & Risikoidentifikation & D1 & D1.2 & FR1, FR2 \\
K1.2 & Risikobewertung & D1 & D1.3 & FR2, FR3 \\
K1.3 & Risikobehandlung & D1 & D1.4 & FR2, FR4 \\
K1.4 & Testverfahren & D1 & D1.5 & FR2 \\
K1.5 & Kont. Aktualisierung & D1 & D1.6 & FR2 \\
K1.6 & RM-Governance & D1 & D1.1 & FR1, FR2 \\
\hline
K2.1 & Datenqualität & D2 & D2.1 & FR1, FR2 \\
K2.2 & Bias-Erkennung & D2 & D2.2 & FR2 \\
K2.3 & Datenherkunft & D2 & D2.3 & FR2 \\
K2.4 & DSGVO-Integration & D2 & D2.4 & FR2 \\
K2.5 & Kont. Datenqualität & D2 & D2.5 & FR2 \\
\hline
K3.1 & Techn. Dokumentation & D3 & D3.1 & FR1, FR2 \\
K3.2 & Logging & D3 & D3.2 & FR2 \\
K3.3 & Versionierung & D3 & D3.3 & FR2 \\
K3.4 & Konformitätsnachweise & D3 & D3.4 & FR2 \\
K3.5 & Dokumentationsmgmt. & D3 & D3.5 & FR2 \\
\hline
K4.1 & Erklärbarkeit & D4 & D4.1 & FR1, FR2 \\
K4.2 & Informationspflichten & D4 & D4.2 & FR2 \\
K4.3 & Fähigk. u. Grenzen & D4 & D4.3 & FR2 \\
K4.4 & KI-Kennzeichnung & D4 & D4.4 & FR2 \\
K4.5 & Prozesstransparenz & D4 & D4.5 & FR2 \\
\hline
K5.1 & Aufsichtsdesign & D5 & D5.1 & FR1, FR2 \\
K5.2 & Qualifikation & D5 & D5.2 & FR2 \\
K5.3 & Intervention & D5 & D5.3 & FR2 \\
K5.4 & Automation Bias & D5 & D5.4 & FR2 \\
K5.5 & Aufsichtsdoku. & D5 & D5.5 & FR2 \\
\hline
K6.1 & Genauigkeit & D6 & D6.1 & FR1, FR2 \\
K6.2 & Robustheit & D6 & D6.2 & FR2 \\
K6.3 & Cybersicherheit & D6 & D6.3 & FR2 \\
K6.4 & Drift-Monitoring & D6 & D6.4 & FR2 \\
K6.5 & Fallback-Mech. & D6 & D6.5 & FR2 \\
\hline
Q1 & Org. Verankerung & D1--D6 & Querschnitt & NFR1, NFR4, NFR5 \\
Q2 & Kompetenzentwickl. & D1--D6 & Querschnitt & NFR1, NFR4, NFR5 \\
\hline
\end{tabular}
\caption{Anforderungsmatrix: Zuordnung Kategorien zu Dimensionen, Kriterien und Requirements}
\label{tab:anforderungsmatrix}
\end{table}

Die Nummerierung der Subkategorien (K\textit{x.y}) und der Bewertungskriterien (D\textit{x.y}) folgt jeweils einer eigenständigen Logik: Die K-Subkategorien sind nach inhaltlicher Erschließungsreihenfolge der Inhaltsanalyse nummeriert, die D-Kriterien nach ihrer konzeptionellen Stellung innerhalb der jeweiligen Governance-Dimension. Eine 1:1-Entsprechung der Indizes (z.\,B. K1.1~$\leftrightarrow$~D1.1) war nicht beabsichtigt; die tatsächliche Zuordnung ergibt sich aus der inhaltlichen Analyse und ist in der obigen Matrix dokumentiert.

Die Begründung jeder Zuordnung ergibt sich aus den Ergebnissen der qualitativen Inhaltsanalyse (Kapitel~\ref{chap:stand}) und ist im Kodierleitfaden (Anhang~\ref{app:kodierleitfaden}) dokumentiert.
