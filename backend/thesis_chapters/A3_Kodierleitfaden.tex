% Anhang A: Kodierleitfaden und Kategoriensystem der Inhaltsanalyse

\addchap{Anhang A: Kodierleitfaden und Kategoriensystem}
\label{app:kodierleitfaden}

Der folgende Kodierleitfaden dokumentiert das Kategoriensystem der qualitativen Inhaltsanalyse, das zur systematischen Erschließung der regulatorischen Anforderungen des EU AI Acts und der Governance-Literatur verwendet wurde. Für jede (Sub-)Kategorie sind Definition, Ankerbeispiele und Kodierregeln angegeben.

\section*{Hauptkategorien (deduktiv)}

\begin{table}[htbp]
\centering
\small
\renewcommand{\arraystretch}{1.3}
\begin{tabular}{|p{1.2cm}|p{3cm}|p{5cm}|p{4cm}|}
\hline
\textbf{Kat.} & \textbf{Bezeichnung} & \textbf{Definition} & \textbf{Ankerbeispiel} \\
\hline
K1 & Risikomanagement & Aussagen zu Risikoidentifikation, -bewertung, -behandlung und -monitoring von KI-Systemen & ``Providers shall establish, implement, document and maintain a risk management system'' (Art.~9 Abs.~1) \\
\hline
K2 & Data Governance & Aussagen zu Datenqualität, Bias-Erkennung, Datenherkunft und Datenschutz-Compliance & ``Training, validation and testing data sets shall be subject to data governance'' (Art.~10 Abs.~2) \\
\hline
K3 & Dokumentation & Aussagen zu technischer Dokumentation, Logging, Versionierung und Konformitätsnachweisen & ``The technical documentation shall be drawn up before the system is placed on the market'' (Art.~11 Abs.~1) \\
\hline
K4 & Transparenz & Aussagen zu Erklärbarkeit, Informationspflichten, Kennzeichnung und Leistungsgrenzen & ``High-risk AI systems shall be designed to ensure that their operation is sufficiently transparent'' (Art.~13 Abs.~1) \\
\hline
K5 & Menschliche Aufsicht & Aussagen zu Human-in/on-the-loop, Aufsichtskompetenzen, Interventionsmechanismen & ``High-risk AI systems shall be designed so that they can be effectively overseen by natural persons'' (Art.~14 Abs.~1) \\
\hline
K6 & Technische Robustheit & Aussagen zu Genauigkeit, Robustheit, Cybersicherheit und Monitoring & ``High-risk AI systems shall achieve an appropriate level of accuracy, robustness and cybersecurity'' (Art.~15 Abs.~1) \\
\hline
\end{tabular}
\caption{Hauptkategorien des Kodierleitfadens}
\label{tab:hauptkategorien}
\end{table}

\section*{Querschnittskategorien (induktiv)}

\begin{table}[htbp]
\centering
\small
\renewcommand{\arraystretch}{1.3}
\begin{tabular}{|p{1.2cm}|p{3cm}|p{5cm}|p{4cm}|}
\hline
\textbf{Kat.} & \textbf{Bezeichnung} & \textbf{Definition} & \textbf{Ankerbeispiel} \\
\hline
Q1 & Organisationale Verankerung & Aussagen zu Rollen, Verantwortlichkeiten, Governance-Strukturen und organisatorischer Einbettung & ``Build on existing policies and governance structures'' (Mökander et al. 2022) \\
\hline
Q2 & Kompetenzentwicklung & Aussagen zu Schulung, AI Literacy, Change Management und Awareness & ``Empower employees through continuous education'' (Mökander et al. 2022) \\
\hline
\end{tabular}
\caption{Induktiv identifizierte Querschnittskategorien}
\label{tab:querschnittskategorien}
\end{table}

\section*{Kodierregeln}

\begin{description}
    \item[Kodiereinheit:] Einzelner Satz oder zusammenhängender Absatz
    \item[Kontexteinheit:] Jeweiliger Artikel des EU AI Acts bzw. Abschnitt der wissenschaftlichen Quelle
    \item[Auswertungseinheit:] Gesamtes Materialkorpus (Normtext EU AI Act, Art.~9--15 inkl. Erwägungsgründe; 33 wissenschaftliche Quellen)
    \item[Mehrfachkodierung:] Zulässig, wenn eine Textpassage mehreren Kategorien zugeordnet werden kann
    \item[Zuordnungsregel:] Im Zweifelsfall wird die spezifischere Kategorie bevorzugt; bei Gleichrangigkeit erfolgt eine Mehrfachkodierung
\end{description}

\section*{Subkategorien mit Ankerbeispielen}

Die folgenden Tabellen dokumentieren die vollständigen Subkategorien für jede Hauptkategorie mit Definitionen und Ankerbeispielen.

\begin{table}[htbp]
\centering
\small
\renewcommand{\arraystretch}{1.3}
\begin{tabular}{|p{1.2cm}|p{3cm}|p{4.5cm}|p{4cm}|}
\hline
\textbf{Subkat.} & \textbf{Bezeichnung} & \textbf{Definition} & \textbf{Ankerbeispiel} \\
\hline
K1.1 & Risikoidentifikation und -analyse & Aussagen zur systematischen Erfassung bekannter und vorhersehbarer Risiken & ``identify and analyse the known and the reasonably foreseeable risks'' (Art.~9 Abs.~2 lit.~a) \\
\hline
K1.2 & Risikobewertungsmethodik & Aussagen zur Bewertung und Quantifizierung identifizierter Risiken & ``estimation and evaluation of the risks that may emerge'' (Art.~9 Abs.~2 lit.~b) \\
\hline
K1.3 & Risikobehandlung & Aussagen zu Maßnahmen der Risikominderung und Restrisiko-Akzeptanz & ``adoption of appropriate and targeted risk management measures'' (Art.~9 Abs.~2 lit.~d) \\
\hline
K1.4 & Testverfahren & Aussagen zu Tests und Validierungsmaßnahmen zur Risikoüberprüfung & ``testing [...] shall be made against prior defined metrics and probabilistic thresholds'' (Art.~9 Abs.~7) \\
\hline
K1.5 & Kontinuierliche Aktualisierung & Aussagen zum lebenszyklusübergreifenden Risiko-Monitoring & ``iterative process [...] throughout the entire lifecycle'' (Art.~9 Abs.~3) \\
\hline
K1.6 & Risikomanagement-Governance & Aussagen zu Verantwortlichkeiten und Organisationsstrukturen im Risikomanagement & ``Establish clear ownership and accountability structures'' (Mökander et al. 2022) \\
\hline
\end{tabular}
\caption{Subkategorien K1 -- Risikomanagement}
\label{tab:subkat-k1}
\end{table}

\begin{table}[htbp]
\centering
\small
\renewcommand{\arraystretch}{1.3}
\begin{tabular}{|p{1.2cm}|p{3cm}|p{4.5cm}|p{4cm}|}
\hline
\textbf{Subkat.} & \textbf{Bezeichnung} & \textbf{Definition} & \textbf{Ankerbeispiel} \\
\hline
K2.1 & Datenqualitätsstandards & Aussagen zu Qualitätsanforderungen an Trainings-, Validierungs- und Testdaten & ``training, validation and testing data sets shall be relevant, sufficiently representative'' (Art.~10 Abs.~3) \\
\hline
K2.2 & Bias-Erkennung und -Korrektur & Aussagen zur Identifikation und Behandlung von Verzerrungen in Datensätzen & ``examination in view of possible biases that are likely to affect'' (Art.~10 Abs.~2 lit.~f) \\
\hline
K2.3 & Datenherkunft und Lineage & Aussagen zur Nachverfolgbarkeit der Datenherkunft und -verarbeitung & ``data sets shall be subject to data governance and management practices'' (Art.~10 Abs.~2) \\
\hline
K2.4 & DSGVO-Integration & Aussagen zur Verknüpfung von KI-Datenverarbeitung und Datenschutzrecht & ``processing of special categories of personal data [...] strictly necessary for the purpose of ensuring bias detection'' (Art.~10 Abs.~5) \\
\hline
K2.5 & Kontinuierliche Datenqualität & Aussagen zur laufenden Datenqualitätssicherung im Betrieb & ``appropriate data governance [...] relevant also to data that will be used after placing on the market'' (Art.~10 Abs.~6) \\
\hline
\end{tabular}
\caption{Subkategorien K2 -- Data Governance}
\label{tab:subkat-k2}
\end{table}

\begin{table}[htbp]
\centering
\small
\renewcommand{\arraystretch}{1.3}
\begin{tabular}{|p{1.2cm}|p{3cm}|p{4.5cm}|p{4cm}|}
\hline
\textbf{Subkat.} & \textbf{Bezeichnung} & \textbf{Definition} & \textbf{Ankerbeispiel} \\
\hline
K3.1 & Technische Dokumentation & Aussagen zur Systembeschreibung und Designdokumentation & ``the technical documentation shall be drawn up before [...] placed on the market'' (Art.~11 Abs.~1) \\
\hline
K3.2 & Automatisiertes Logging & Aussagen zur automatischen Aufzeichnung von Systemereignissen & ``automatically record events (logs) [...] to ensure a level of traceability'' (Art.~12 Abs.~1) \\
\hline
K3.3 & Versionierung und Änderungskontrolle & Aussagen zur Nachverfolgung von Systemänderungen & ``kept up to date'' und ``updated where necessary'' (Art.~11 Abs.~1, Art.~9 Abs.~3) \\
\hline
K3.4 & Konformitätsnachweise & Aussagen zur Dokumentation der regulatorischen Konformität & ``shall be sufficient to demonstrate that the high-risk AI system complies'' (Art.~11 Abs.~1) \\
\hline
K3.5 & Dokumentationsmanagement & Aussagen zur Organisation und Verwaltung der Dokumentation & ``documentation [...] shall contain at least the information set out in Annex~IV'' (Art.~11 Abs.~1) \\
\hline
\end{tabular}
\caption{Subkategorien K3 -- Dokumentation}
\label{tab:subkat-k3}
\end{table}

\begin{table}[htbp]
\centering
\small
\renewcommand{\arraystretch}{1.3}
\begin{tabular}{|p{1.2cm}|p{3cm}|p{4.5cm}|p{4cm}|}
\hline
\textbf{Subkat.} & \textbf{Bezeichnung} & \textbf{Definition} & \textbf{Ankerbeispiel} \\
\hline
K4.1 & Erklärbarkeit & Aussagen zur Nachvollziehbarkeit der KI-Entscheidungsfindung & ``sufficiently transparent to enable deployers to interpret the system's output'' (Art.~13 Abs.~1) \\
\hline
K4.2 & Informationspflichten & Aussagen zu Pflichten der Informationsbereitstellung gegenüber Nutzern & ``accompanied by instructions for use [...] that include concise, complete, correct and clear information'' (Art.~13 Abs.~2) \\
\hline
K4.3 & Fähigkeiten und Grenzen & Aussagen zur Kommunikation von Systemfähigkeiten und -einschränkungen & ``the level of accuracy [...] and the foreseeable circumstances that may have an impact'' (Art.~13 Abs.~3 lit.~b) \\
\hline
K4.4 & KI-Kennzeichnung & Aussagen zur Kennzeichnung als KI-System & ``natural persons shall be informed that they are interacting with an AI system'' (Art.~50 Abs.~1) \\
\hline
K4.5 & Prozesstransparenz & Aussagen zur Transparenz der Governance-Prozesse selbst & ``Framework visibility and organizational communication'' (Ryan \& Stahl 2020) \\
\hline
\end{tabular}
\caption{Subkategorien K4 -- Transparenz}
\label{tab:subkat-k4}
\end{table}

\begin{table}[htbp]
\centering
\small
\renewcommand{\arraystretch}{1.3}
\begin{tabular}{|p{1.2cm}|p{3cm}|p{4.5cm}|p{4cm}|}
\hline
\textbf{Subkat.} & \textbf{Bezeichnung} & \textbf{Definition} & \textbf{Ankerbeispiel} \\
\hline
K5.1 & Aufsichtsdesign & Aussagen zur Gestaltung der menschlichen Aufsicht (HITL/HOTL) & ``designed and developed in such a way [...] that they can be effectively overseen by natural persons'' (Art.~14 Abs.~1) \\
\hline
K5.2 & Qualifikation der Aufsichtspersonen & Aussagen zur Kompetenz und Befähigung von Aufsichtspersonen & ``individuals assigned for human oversight are specifically aware of the possible confirmation bias'' (Art.~14 Abs.~4 lit.~b) \\
\hline
K5.3 & Interventionsmechanismen & Aussagen zu Eingriffsrechten und -möglichkeiten & ``able to decide [...] not to use the system, to override or reverse the output'' (Art.~14 Abs.~4 lit.~d--e) \\
\hline
K5.4 & Prävention von Automation Bias & Aussagen zur Vermeidung übermäßigen Vertrauens in KI-Ausgaben & ``minimise the risk of automation bias'' (Art.~14 Abs.~4 lit.~b) \\
\hline
K5.5 & Aufsichtsdokumentation & Aussagen zur Dokumentation und Überprüfung der Aufsichtstätigkeit & ``effective oversight by natural persons during the period of use'' (Art.~14 Abs.~1) \\
\hline
\end{tabular}
\caption{Subkategorien K5 -- Menschliche Aufsicht}
\label{tab:subkat-k5}
\end{table}

\begin{table}[htbp]
\centering
\small
\renewcommand{\arraystretch}{1.3}
\begin{tabular}{|p{1.2cm}|p{3cm}|p{4.5cm}|p{4cm}|}
\hline
\textbf{Subkat.} & \textbf{Bezeichnung} & \textbf{Definition} & \textbf{Ankerbeispiel} \\
\hline
K6.1 & Genauigkeitsmetriken & Aussagen zur Messung und Sicherstellung der Systemgenauigkeit & ``achieve an appropriate level of accuracy'' (Art.~15 Abs.~1) \\
\hline
K6.2 & Robustheitstests & Aussagen zu Tests gegen Störungen und adversariale Eingaben & ``resilient against attempts by unauthorised third parties to alter their use, outputs or performance'' (Art.~15 Abs.~4) \\
\hline
K6.3 & Cybersicherheit & Aussagen zu IT-Sicherheitsmaßnahmen für KI-Systeme & ``an appropriate level of [...] cybersecurity'' (Art.~15 Abs.~1) \\
\hline
K6.4 & Drift-Erkennung und Monitoring & Aussagen zur Erkennung von Modellverschlechterungen im Betrieb & ``take into account the reasonably foreseeable conditions'' (Art.~15 Abs.~5) \\
\hline
K6.5 & Fallback-Mechanismen & Aussagen zu Rückfallstrategien bei Systemversagen & ``appropriate mitigation measures [...] in the event of an incident'' (abgeleitet aus Art.~9 Abs.~2 lit.~d i.\,V.\,m. Art.~15) \\
\hline
\end{tabular}
\caption{Subkategorien K6 -- Technische Robustheit}
\label{tab:subkat-k6}
\end{table}

\section*{Abgrenzungsregeln für Grenzfälle}

\begin{description}
    \item[K1 vs. K6:] Textstellen, die sowohl Risikomanagement als auch technische Robustheit adressieren (v.\,a. bei Art.~15), werden K6 zugeordnet, wenn der technische Aspekt (Genauigkeit, Robustheit, Sicherheit) dominiert. Übergeordnete Prozessanforderungen (Managementsystem, Iteration) werden K1 zugeordnet.
    \item[K2 vs. K4:] Datenschutz- und Datentransparenzanforderungen werden K2 zugeordnet, wenn sie die Datenverarbeitung betreffen, und K4, wenn sie die Informationsweitergabe an Nutzer oder Betroffene adressieren.
    \item[K3 vs. K1/K6:] Dokumentationsanforderungen, die sich auf spezifische Governance-Maßnahmen beziehen (z.\,B. Dokumentation des Risikomanagements), werden der inhaltlichen Kategorie (K1) zugeordnet. Übergreifende Dokumentationsanforderungen (technische Dokumentation, Logging) werden K3 zugeordnet.
    \item[Q1/Q2 vs. K1--K6:] Textstellen werden den Querschnittskategorien zugeordnet, wenn sie organisationale Voraussetzungen oder Kompetenzen als eigenständigen Governance-Aspekt beschreiben -- nicht als Teilaspekt einer spezifischen regulatorischen Anforderung.
\end{description}
